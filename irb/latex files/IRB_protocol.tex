% \documentclass[12pt,a4paper]{article}
\documentclass[12pt]{article}
\usepackage[left=1in, right=1in, top=1in, bottom=1in, includehead=false, includefoot=false]{geometry}
\usepackage{url}

\usepackage[utf8]{inputenc}
\usepackage{caption}
\usepackage{subcaption}
\usepackage{natbib}
\usepackage{epsfig,graphicx}
\usepackage{float}
\usepackage{chngpage}
\usepackage{multirow}
\usepackage[hidelinks]{hyperref}
\usepackage[none]{hyphenat}
\usepackage[all]{nowidow}
\usepackage[dvipsnames]{xcolor}
\usepackage{bm}
\usepackage{soul}
\usepackage{textcomp}

\usepackage{times}
\usepackage{xcolor}

\urlstyle{rm}

% \usepackage{float}
% \restylefloat{table}

\bibliographystyle{chicago} % agsm
% \bibliographystyle{../style/chicago}

\graphicspath{{../figures/}}

\captionsetup{%
   justification=raggedright
}

% \usepackage[doublespacing]{setspace}
% \usepackage[document]{ragged2e}
% \setlength{\RaggedRightParindent}{\parindent}

\setlength{\bibsep}{0pt plus 0.5ex}
\newlength{\li} \setlength{\li}{12pt}
\newlength{\si} \setlength{\si}{.4in}
\setlength{\parindent}{\si}
\setlength{\parsep}{0pt}

\makeatletter
\newcommand\iraggedright{%
  \let\\\@centercr\@rightskip\@flushglue \rightskip\@rightskip
  \leftskip\z@skip}
\makeatother

\newcommand{\redtxt}{\textcolor{red}}

\date{}



\title{How Well Computers Can Read Spatial Patterns using Visual Symbols in Choropleth Maps Compared with Humans}

\begin{document}
\iraggedright
\maketitle

\section{Study Objectives}
The goal of this study is to explore how well computers with artificial intelligence (AI) algorithms can read spatial patterns using visual symbols in choropleth maps compared with humans. An example of choropleth maps in our survey is shown above. Visual symbols in this study mean the colors of enumeration units (e.g., states in a state-level population density map for the U.S.) in choropleth maps. The survey to conduct in this protocol will be used to collect the data for the performance of humans on reading spatial patterns of choropleth units, which will be compared with computers in further analysis.

\section{Background and Rationale}
A map is the graphic representation of the environment \citep{robinson1995elements}. With the development of free map making tools, making digital maps are more accessible than ever \citep{Robinson2019}. The Internet represents a new medium for cartography with the seemingly limitless range of online map products available \citep{peterson2005maps}. The increasing quality and availability of the digital maps is now providing a realistic alternative to the traditional paper map format \citep{Hurst2013}. An increasing proportion of mapping contents are being presented and consumed by use of a digital interface of some description \citep{dodge2011mapping}. For example, the famous traditional media such as New York Times and Forbes use digital maps on the websites to help convey their opinions. Digital maps including online map images have evolved in recent years to become a part of our everyday life \citep{Aly2017}. 

There are different types of online maps including thematic maps to display a spatial phenomenon (i.e., theme) using different visual symbols. Thematic maps can be further divided into categories such as choropleth maps. In the research, we focus on choropleth maps, which are a type of the most commonly-used thematic maps. In a choropleth map, enumeration units (such as states in a state-level choropleth map of the United States) are filled with different colors, which are the visual symbols used in choropleth maps. Each color represents a value range of the measurement of a map theme such as population density \citep{slocum2009thematic}. The colors in choropleth maps show different spatial patterns of the presented phenomenon. Spatial pattern is the description for the arrangement or placement of geographic features \citep{kimerling2016map} such as how the trees in the Oval on the Ohio State campus are distributed. We select two common spatial patterns in this study for computers and humans to read:(1) whether the represented phenomenon concentrated or clustered in some area and (2) whether the values tend to occur near their similar vlaues or different values. 

Humans develop map reading skills \citep{presson1982development,gilhooly1988skill} to comprehend qualitative and quantitative information from the map naturally, because maps are artifacts that are made by humans and, more importantly in this context, {\it for} humans to read. But maps are not designed for computers to read, so it can be difficult for computers to read and comprehend maps. In this protocol, reading maps by computers means computers can tell us what is on the map, for example, whether the represented phenomenon concentrated or clustered in some area. Computers operate on a different system and may need a leap of faith to read maps. Nevertheless, intelligent machines that can understand artifacts (including maps) seem to be trending toward being part of our future \citep{cascio2016technology} and exploring such a future in map making is interesting and exciting. 

Reading maps has been challenging, if not impossible, for traditional computer algorithms. However, dramatic progress has been made in computer vision in the last decade, thanks to the advances in artificial intelligence such as deep neural networks \citep{Szeliski2021}. Methods in computer vision have been successfully applied to problems in various domains where images need to be recognized or classified. For example, in facial recognition, deep learning models are applied on a face photograph database designed for studying the problem of unconstrained face recognition and have achieved an accuracy that approaches or is even beyond humans \citep{Wang2021}. In autonomous vehicle detection, deep learning models are used to enable self-driving cars to recognize different objects, such as vehicles, traffic signals, and pedestrians on the road \citep{Sang2018,Wang2019}. In medical science, methods such as support vector machine (SVM) and convolutional neural networks (CNNs) have also been successfully used to diagnose diseases from tissue images, where these methods can achieve even higher diagnosis accuracy than human doctors \citep{Madabhushi2016}. However, there is few research to recognize and analyze contents in maps. 

There is no doubt that artificial intelligence has not reached a point that machines can read maps as humans do. But it is possible that computers can recognize some contents of maps utilizing the cutting-edge algorithms in computer vision and then do some analysis such as recognizing spatial patterns presented in the maps. In this study, we focus on reading spatial patterns using visual symbols (colors in mapping area) in choropleth maps, which is a relatively simple and natural tasks for humans but challenging for computers. The deep learning-based object detection models in computer vision can be used to detect the mapping area of a choropleth map. Then, algorithms in traditional computer vision, data mining, and other AI algorithms can be applied to extracting more information from the mapping area and analyzing its spatial patterns. We aim to explore how to utilize the AI techniques to recognize spatial patterns in choropleth maps with computers and and further analyze how to understand how well the computers can do compared with humans. So, we design questions about spatial patterns in choropleth maps for both computers and humans to answer. In this protocol, we only discuss how to conduct surveys with humans as subjects to read spatial patterns in choropleth maps.

\section{Procedures}

\subsection{Research design}

This is a study to measure how well humans can read spatial patterns using visual symbols in choropleth maps. Participants will be asked to answer questions in the survey in the platform of Qualtrics. Qualtrics is a market leader and top rated digital survey software. With its strong survey functionality including flexible workflow control and privacy protection, researchers have used it as a survey tool to collect and analyze many types of survey data. 

The questions are about spatial patterns of enumeration units for classes in choropleth maps, including concentration and scattering condition and spatial autocorrelation of the units. Spatial autocorrelation means that the same classes are more likely to occur in the neighborhood of the same classes (positive autocorrelation) or less likely to occur in the neighborhood of the same classes (negative autocorrelation). More specifically, we ask two questions: (1) Is the phenomenon represented by the visual symbols (colors) concentrated in some area scattered over the map? (2) Do the values tend to occur near their similar values or different values? The answers from our subjects will be collected from Qualtrics survey reports and analyzed by us comparing with the performance of computers.

The difficulty to read choropleth maps can be affected by different variables. We use maps with different design variables in the survey to test the human performance for different levels of reading difficulties. In order for better control of the design variables, we synthesize choropleth maps for testing.
The synthetic maps are not normal map as we see in daily lives. The purpose of this experiment is to understand how visual symbols (colors) can be used to identify spatial patterns. So, we removed titles, legend headings, only the visual symbols are presented.

Different base maps with different number and sizes of enumeration units are the first variable for the reading difficulty. Because the number and size of enumeration units can influence the reading difficulties of humans. If the relative size of enumeration units is small and there are many units in mapping area, it is more difficult for humans to process the information. If there are only a few enumeration units with large sizes, humans should be easy to recognize the presented information. Therefore, in our study, we adopt two base maps: maps for Ohio at county level (easy) and maps for contiguous United States at county level (difficult). 

The second design variable is the color scheme. We apply color schemes suggested by ColorBrewer\footnote{https://colorbrewer2.org/} to color the enemeration units in the two base maps. ColorBrewer is a simple online tool that helps map designers choose appropriate color schemes to use in their maps. We synthesize maps with sequential and diverging color schemes. For each color scheme, we  select 3 different numbers (i.e., 4, 6, and 8) of colors used in one map. The types of color schemes and number of colors can also influence the difficulties of map reading. For example, it is easier to recognize relationships between high and low values if a diverging color scheme is used. And when there are many colors (>=8) used in one map, map users will find it difficult to read even though a sequential or diverging color scheme is used.

The spatial data used in synthetic maps present different spatial patterns for our subjects to recognize. We synthesize spatial data for the two base maps with four types of spatial autocorrelation: negative spatial autocorrelation, no spatial autocorrelation, small positive spatial autocorrelation, and large positive spatial autocorrelation.

Considering the design variables and different spatial patterns in maps, we will synthesize 2 (base map) x 2 (color scheme type) x 3 (number of colors) x 4 (spatial patterns) x 2 = 96 maps in total, which will cover twice of all combinations of the design variables and spatial patterns for maps. In the survey, we will allocate 16 map images with two questions for each to one participant. It will take one participant about one minute for one question. They will spend about 20 minutes on the survey, which is not too long to discourge more subjects to participate in our survey. 

\subsection{Sample}

The project outlined here is meant to service as the human map reading ability survey to compare with how well computers can read spatial patterns using visual symbols in choropleth maps compared with humans. We aim to establish some baseline understanding of how well humans can read spatial patterns in choropleth maps with the two questions above. As such, we do not need a large sample size. We will recruit volunteers in the academia, especially those with background in geography and/or using maps to portray spatial data. We plan to invite approximately 100 adult (age 18-65) from Ohio state students, faculty, and staff members to answer the questions in our survey. The survey will be distributed in two undergraduate classes in the Department of Geography: Design and Implementation of GIS (GEOG 5223) and GeoVisualization (GEOG 5201) as extra credits for homework. We will use online bulletin boards, email mailing lists, in-class announcements, and personal contacts to reach out to potential participants.

\subsection{Measurement/instrumentation}

Measurement: The experiments will evaluate the reading performance of participants on spatial patterns in choropleth maps using choropleth maps. For each question, the participants will be asked to select one from the chioces. The answers will be saved in a survey report of the Qualtrics platform. Then, the accuracy rates for the questions of maps will be calculated. The accuracy rates will be compared with rates by computers.

Instrumentation: To conduct the experiments, each participant needs a computer system (including a monitor and a mouse). All experiments will be done online through using a web browser. The application can start the experiments at any time or at any place without log-in requirements. Data will be saved on the server of Qualtrics, which can be retrived for further analysis. All the data collected here is anonymous, and no identifiable information will be recorded. The surveys are conducted on the platform of Qualtrics, and the choropleth map images are synthesized by Python. 


\subsection{Detailed study procedures}

Each participant will be encouraged to take the experiments in a safe environment with relatively low ambient noise, which can be disburbing during the survey. The questions for each participant are called one experiment.   

Each experiment contains four sections: informed consent form, instruction, background information, question, and random completion code generation for undergraduate students from two geography classes.

\begin{itemize}
    \item The informed consent form explains the the goal of the study, procedures and tasks, risks and benefits, confidentiality, and participant rights. Our participants will have a basic understanding of this project, then they can decide to continue the survey or not. 
    
    \item The instruction section introduces what the participants will see and what they should do during the survey. An example of a map with the two questions is included in the instruction. After this section, the participant should know what to do during the experiment. 

    \item In the section of backgroud information, the participant will be asked to answer their background in geography and cartography, which will be used for experiment control purposes.

    \item In the question section, the participant will be presented choropleth maps with different spatial patterns and be asked the two questions about spatial pattern for each map. They should read the maps carefully and answer the questions based on what they read and learn from the mapping areas.

    \item At last, there will be a random completion code generated, which is only for students in the two geography classes (GEOG 5223 and GEOG 5201) mentioned above. The completion code will be submited in his or her Carmen system, serving as the proof of completion of extra homework. After copying the completion code, a student (and other participants) can submit the survey. In this way, the personal information of student participants can be protected.
\end{itemize}

\subsection{Internal validity}

The goal of our study is to explore how well computers can read spatial patterns in choropleth maps comparing with humans. The survey for humans are designed to measure the ability of humans to read spatial patterns. The two questions about spatial patterns of the phenomenon presented in a map are non-trivial questions. They require humans to read a map especially the mapping area carefully. Only if humans can read the visual symbols, they can analyze the spatial patterns of the classes to answer the questions. The participants will be asked to take the survey in a relatively quiet environment without disturbs. And the devices especially monitors should be of good quality to display choropleth maps with no significant color differences. The performance of humans on the two questions will be compared with the performance of computers with AI algorithms.

The experiment will not discriminate against any population and will be open
to all on a voluntary basis. This experiment utilizes a forced choice response questions in which the correct choices in the experiments are predefined by the researchers. We will also make some slides with instructions to introduce the experiments and how the Qualtrics survey will work. By using the instructions, we expect the participants will be familiar with the general process of the experiments and then concentrate on the experiments when they start.


\subsection{Data handling analysis}

Data would be collected by asking participants to read maps and answer questions. Key personal information such as name and personal email would not be collected. The random completion code is generated by the Qualtrics platform and this code does not contain any personal information. Thus, we as researchers would not be able to know the workers' key personal information such as name and cannot identify who they are behind the screen. Thus, subjects are actually totally anonymous during the whole research process and subject confidentiality would therefore be ensured throughout the whole experiment.

The average correctness rate of the responses with respect to different base maps and color schemes will be used to reveal the impacts of the variables on the reading performance of our subjects. The performance of human performance on different spatial patterns will also be analyzed. Further, we will develop statistical models to understand the relationship between the human reading performance on choropleth maps and the map design variables and spatial patterns. 
Examining the difference of correctness of participants` responses
between the two base maps (Ohio map at county level and continuous U.S. map at county level) will reveal the impact of the number and size of enumeration units on recognizing spatial patterns by humans. Similar analysis can be conducted to understand the relationship between human map reading performance and colors (types of color schemes and number of colors) used in maps. Then, the human performance for recognizing spatial patterns of choropleth classes will be compared with computer performance.

\section{bibliography}

% \printbibliography %Prints bibliography
\bibliography{references}


\end{document}